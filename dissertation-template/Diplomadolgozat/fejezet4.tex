%----------------------------------------------------------------------------
\chapter{A rendszer specifikációja}
%----------------------------------------------------------------------------



Ajánlott a \href{https://drive.google.com/file/d/1Fmg_N5Zov1e_GZzF8uGK2GAkMDoo0_vm/view?usp=sharing}{Volere template} átnézése: egyrészt ad egy keretet, másrészt rengeteg jó kérdést tesz fel, melyre csak válaszolni kell a mi projektünk esetében. Azokat a részeket, amelyek nem relevánsak, egyszerűen ki lehet hagyni.
Az UML diagramok készítésénél ajánlott a James Sugrue \'altal k\'esz\'itett útmutatót használni\footnote{\href{https://drive.google.com/drive/folders/1ocZsC26FMHKStwkEhZx8H4od6Mk3Deo3}{\'Utmutat\'o}}.


\section{Felhasználói követelmények}


Ide jönnek a használati esetek (use case diagram), és a hozzá tartozó leírás. Figyelem: itt csak felhasználói szemszögből tekintünk a rendszerre. A felhasználó nem ismeri a Python programozási nyelvet, sem a publikus kulcsszót. 






\section{Rendszerkövetelmények}

Itt már rendszer szemszög van előtérben. 

\begin{enumerate}
	\item[a)] \textbf{Funkcionális követelmények, mire képes a rendszer.} A fontosabb funkcionalitásokat egyenként kell részletezni. Erre használható az \href{https://www.businessanalystlearnings.com/blog/2013/9/30/the-3-step-guide-to-documenting-requirements-with-use-cases}{itt} látható sablon. Lehet aktivitás-, szekvencia-, illetve együttműködési diagramokat hasznáni.
	\item[b)] \textbf{Nem funkcionális követelmények, megszorítások.} Itt érdemes kiindulni a \href{https://cs.ccsu.edu/~stan/classes/CS410/notes16/04-Requirements.html}{Sommerville} csoportosításból: 
	
	\begin{itemize}
		\item Termékkel kapcsolatos követelmények (pl mennyi tárhelyre van szükség?), 
		\item Hány kérést tud kiszolgálni másodpercenként, 
		\item Szervezéssel kapcsolatos követelmények (pl. Milyen operációs rendszer, vagy milyen verziókövető rendszer?),
		\item Külső követelmények (pl GDPR).
	\end{itemize}
	
	
	
	
	
	
	
\end{enumerate}






\section {P\'elda}

A \ref{fejezet3}. fejezetben bemutattam két beépített, előre megvalósított szoftvert, melyek a Matlab ode45 programja és a Boost - Odeint könyvtár. Az elért eredmények és tesztek azt mutatják, hogy a Matlab ode45 differenciálegyenlet megoldója sokkal gyorsabb és hatékonyabb, mint az Odeint. Ha a tesztesetekben mért átlagidőket összehasonlítjuk láthatjuk (\ref{fejezet3_3}. alfejezet), hogy az ode45 $ 20-30 $ -szor gyorsabb a Odeintnél. Ez talán annak is köszönhető, hogy a Matlab egy nagyon komoly szoftver, amelynek programcsomagjain mérnökök és programozók százai  (vagy akár ezrei) dolgoznak, így természetes, hogy az algoritmusok jobban optimizáltak és hatékonyabbak az ingyenes szoftvereknél. A továbbiakban összegezzük, hogy a két technológiának milyen előnyei és hátrányai vannak vagy éppen miért érdemes/nem érdemes használni őket, l\'asd \cite{Lovasz}
\\ \\
\textbf{Matlab ode45 előnyei:}
\begin{itemize}
	\item nagyon egyszerű a használata, nem igényel komoly programozási ismereteket
	\item könnyű beépíteni és összekötni más Matlab programokkal
	\item a kapott eredményeket mátrix vagy vektor típusokban téríti vissza, ami könnyűvé teszi az eredmények további kezelését
	\item az eredményeket grafikus felületen azonnal meg tudjuk jeleníteni
	\item jobb eredményeket produkált, mint az Odeint
	\item jól dokumentált, sok példa van a használatára
\end{itemize}
\pagebreak
\textbf{Matlab ode45 hátrányai:}
\begin{itemize}
	\item komoly hátránya az Odeinttel szemben, hogy \textbf{fizetni kell a használatáért}
	\item nagyon sok memóriát használ, komolyan igénybeveszi a számítógép erőforrásait
\end{itemize}
\textbf{Odeint előnyei:}
\begin{itemize}
	\item ingyenes és nyílt forráskódú, használható személyi és kereskedelmi célokra egyaránt
	\item nagyon rugalmas, absztrak, így könnyedén változtatható a bemeneti adatok típusa vagy struktúrája 
	\item C++ nyelven írodott, támogatva a modern programozási technológiákat (Generikus programozás, Template Metaprogramming)
\end{itemize}
\textbf{Odeint hátrányai:}
\begin{itemize}
	\item használata nehezebb, mint a Matlab ode45 programé, szükséges a C++ programozási nyelv ismerete
	\item a kapott eredményekkel nem olyan könnyű bánni, mint a Matlab esetében
	\item absztraktsága miatt nehéz a felmerülő problémákat megoldani
	\item dokumentáltsága jóval szegényesebb, mint a Matlabé
\end{itemize}




\begin{figure}
	\centering
	\begin{tikzpicture}
		\node[main node] (1) {$1$};
		\node[main node] (2) [below left = 2.3cm and 1.5cm of 1]  {$2$};
		\node[main node] (3) [below right = 2.3cm and 1.5cm of 1] {$3$};
		
		\path[draw,thick]
		(1) edge node {} (2)
		(2) edge node {} (3)
		(3) edge node {} (1);
		%%
		\begin{scope}[xshift=4cm]
			\node[main node] (1) {$1$};
			\node[main node] (2) [right = 2cm  of 1]  {$2$};
			\node[main node] (3) [below = 2cm  of 1] {$3$};
			\node[main node] (4) [right = 2cm  of 3] {$4$};
			
			\path[draw,thick]
			(1) edge node {} (2)
			(1) edge node {} (4)
			(3) edge node {} (2)
			(3) edge node {} (4)
			;
		\end{scope}
	\end{tikzpicture}
\caption{Egyszer\H u gr\'af TIKZ seg\'its\'eg\'evel}
\end{figure}




Továbbá megvalósítottam az Euler és Runge-Kutta módszerek párhuzamosított változatait is CUDA technológia segítségével. Ebben az esetben a tesztek azt mutatták, hogy az Euler módszer esetében többe kerül a sok CPU és GPU memória közötti másolás művelete, mint amennyit nyerünk a számítások elvégzése során. Tehát ebben az esetben ez a fajta párhuzamosítási megközelítés nem éri meg. Ezzel ellentétben a Runge-Kutta módszer esetében a megközelítés eredményesnek bizonyult abban az esetben, ha az egyenletek száma nagy és a lépésköz kicsi. A \ref{fejezet3_4} alfejezetben láthattuk, hogy abban az esetben ha az egyenletek száma $ n = 10 $ és a lépésköz $ h = 0.001 $, a GPU-n megközelítőleg 5 és fél perccel hamarabb lefutott az algoritmus, mint a CPU-n. Ezzel a párhuzamosítási módszerrel nem tudtuk kihasználni a videókártya által nyújtott maximális számítási kapacitást, de így is jelentős különbséget sikerült elérni a futási időket nézve, l\'asd \cite{Katai}.


